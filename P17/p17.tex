\documentclass[a4]{article}
\usepackage[utf8]{inputenc}
\usepackage[T1]{fontenc}
\usepackage{a4,graphics,amsmath,amsfonts,amsbsy,amssymb,amsthm}
\usepackage{graphicx}
\usepackage{hyperref}
\usepackage{float}
\usepackage{listings}
\usepackage{enumerate}
\usepackage{comment}
\usepackage{accents}

\setlength{\parskip}{8pt plus 1pt minus 1pt}
%Verdur ad vera her, sumir pakkar dependa a thetta.
\usepackage[icelandic]{babel}

\title{Operating Systems\\
Project 17}
\author{
    Haukur Óskar Þorgeirsson\\
    hth152@hi.is \and
    Matthías Páll Gissurarson\\
    mpg3@hi.is
}
\begin{document}
\maketitle

\section{First attempt}
Let's imagine that all the philosophers get hungry at the same
time. They then all grab the right one first, but as they are sitting
in a circle, the left one will be occupied by the one on their
left. Thus they all wait for the left chopstick, which never becomes
available, as they will not release their right chopstick (which is
the left one of somebody else) until they get the left chopstick
(which is the right one of somebody else). Thus they wait endlessly.
\begin{description}
\item[Mutual exclusion] Each chopstick can only be used by one
  philosopher at a time, so there is Mutual exclusion
\item[Hold and wait] Each philosopher holds the right chopstick and
  waits for the left one that the philosopher to the left is holding
\item[No preemption] Each philosopher releases their chopstick only
  when they have eaten, so there is no preemption
\item[Circular wait] Each one is waitin for the one on their left to
  release their chopstick so if we enumerate them, $P_1$ waits for
  $P_2$, $P_2$ waits for $P_3$, which waits for $P_4$, which waits for
  $P_5$, which waits for $P_1$, so we have Circular wait, and thus we
  have a deadlock.
\end{description}

\section{Second attempt}
Here the situation is much the same as in the first attempt, except
for the fact that there is no circular wait. There can only be a
deadlock when all want to eat at the same time, but then, if we let
$P_1$ be left handed, $P_1$ waits for $P_5$, $P_2$ waits for $P_3$,
$P_3$ waits for $P_4$, $P_4$ waits for $P_5$ which waits for $P_1$. As
there is no circular wait (and cannot be), there is no deadlock.
\end{document}
