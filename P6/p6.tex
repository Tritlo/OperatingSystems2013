\documentclass[a4]{article}

\input{/home/tritlo/Dropbox/Latex/header.tex}


\nonums{}

\title{
    Operating Systems\\
    Project 6
}
\author{
    Haukur Óskar Þorgeirsson\\
    hth152@hi.is\\
    Matthías Páll Gissurarson\\
    mpg3@hi.is
}

\begin{document}
\maketitle

\section{Creating and terminating processes in Windows}

\subsection*{CreateProcess()}

CreateProcess is the Windows system call for creating a process (\emph{surprisingly}). It has the following parameters:

\subsubsection*{bInheritHandles}

A boolean value. Determines whether the new process inherits the calling process' inheritable handles.

\subsubsection*{dwCreationFlags}

The flags that control the proirity class and the creation of the process.

\subsubsection*{lpStartupInfo}

A pointer to a STARTUPINFO structure.

\subsubsection*{lpProcessInformation}

A pointer to which indetification information about the new pointer will be sent.\\\\

\textbf{The following parameters are optional}

\subsubsection*{lpApplicationName}

The name of the module to be executed. E.g. the location of the .exe file to be executed.

\subsubsection*{lpCommandLine}

The command line to be executed.

\subsubsection*{lpProcessAttributes}

Determines whether the returned handle to the new process object can be inherited by child processes.

\subsubsection*{lpThreadAttributes}

Same as above. But for threads.

\subsubsection*{lpEnvironment}

A pointer to the environment block for the new process.

\subsubsection*{lpCurrentDirectory}

The path to the current directory for the process.

\subsection*{TerminateProcess}

The Windows system call for terminating the specified process as well as all of it's threads. It has two parameters:

\subsubsection*{hProcess}

A handle to the process to be terminated.

\subsubsection*{uExitCode}

The exit code to be used by the process terminated and it's threads.

\section{Creating a process in Windows vs. POSIX}

In Windows you create a completely new process each time you call \texttt{CreateProcess}. In POSIX, this can not be done. Sort of. It is possible (and often done) to first use \texttt{fork()} to create a new, duplicate process and then use \texttt{exec()} to replace the current process image of the duplicate process with a new process.

\section*{Sources}

\begin{verbatim}

http://msdn.microsoft.com/en-us/library/windows/desktop/ms682425(v=vs.85).aspx
http://msdn.microsoft.com/en-us/library/windows/desktop/ms686714(v=vs.85).aspx
http://pubs.opengroup.org/onlinepubs/009604499/functions/exec.html
http://pubs.opengroup.org/onlinepubs/9699919799/functions/fork.html

\end{verbatim}

\end{document}
