\documentclass[10pt,a4paper]{article}

\usepackage[T1]{fontenc} 
\usepackage[icelandic]{babel} 
\usepackage[utf8]{inputenc} 
\usepackage{epsfig}
\usepackage{amsmath}
\usepackage{amsfonts}
\usepackage[margin=3cm]{geometry}
\usepackage{listings}
\usepackage{graphicx}
\usepackage[section]{placeins}
%\usepackage[framed,numbered,autolinebreaks,useliterate]{mcode}
\input{kvmacros}

\newcommand{\Nz}{\mathbb{N}_0}
\newcommand{\ilc}{\lstinline}
\newcommand{\fig}[3]{\begin{figure}[h!]
\centering
\includegraphics#1{#2}
\caption{#3}
\end{figure}}
\newcommand{\bbm}{\begin{bmatrix}}
\newcommand{\ebm}{\end{bmatrix}}

\begin{document}

\title{Operating Systems\\Project 18}
\author{Haukur Óskar Þorgeirsson\\hth152@hi.is \and Matthías Páll Gissurarson\\mpg3@hi.is}
\maketitle

\section{}

\fig{[height=6cm]}{base.png}{At the beginning}

\fig{[height=6cm]}{step1.png}{P$_1$ requests R$_1$}

\fig{[height=7cm]}{step2.png}{P$_1$ is assigned R$_1$ and requests R$_2$}

\fig{[height=7cm]}{step3.png}{P$_1$ is assigned an instance of R$_2$ and P$_2$ requests R$_3$}

\fig{[height=7cm]}{step4.png}{P$_2$ is assigned R$_3$ and requests R$_1$}

\fig{[height=7cm]}{step5.png}{P$_2$ is not assigned R$_1$ as P$_1$ has it. P$_1$ requests R$_3$ but will not get it because P$_2$ has it. Neither process can continue as they are both waiting for a resource the other one has but will not release until it has finished. We can also see the cycle clearly, it is a deadlock.}

\section{}

At the beginning, the state is such:

\[
N=M-C=\bbm 1&1&1\\1&1&1 \ebm - \bbm 0&0&0\\0&0&0 \ebm
\]

($C$ is all zeroes as this is the initial state and no resources have been requested or allocated. $M$ describes how each process needs one of each resource.)

\[
A=\bbm 1&2&1 \ebm
\]

Now, \\

$\bbm 1&1&1 \ebm \le \bbm 1&2&1 \ebm$ (pointwise, the former being the first row of $M$)

So we mark P$_1$ and $A:=A+\bbm 0&0&0 \ebm$ (The latter being the first row of $C$). Also,\\

$\bbm 1&1&1 \ebm \le \bbm 1&2&1 \ebm$ (pointwise, the former being the second row of $M$)

So we mark P$_2$ and $A:=A+\bbm 0&0&0 \ebm$ (The latter being the second row of $C$).\\

Both processes have been marked, so this is a safe state. an example og a schedule that would be without deadlocks is:\\

P$_1$: R$_1$; P$_1$: R$_2$; P$_1$: R$_3$; P$_2$: R$_3$; P$_2$: R$_1$; P$_2$: R$_2$.


\end{document}
